\documentclass{CML_Seminar_Template}

\begin{document}

\maketitle{Interruptions in HCI and Boundary Management}
{Sub-title (Times 18; 2 Pt above; flush left)}

\author{First\_name Last\_name (Times 14; 12 Pt above; flush left)}
       {Matrikelnummer (max. 1 line; Times 12; flush left)}
       {Email address (max. 1 line; Times 12; flush left)}

\begin{abstract}
Short abstract, max. 200 words  (Helvetica 10; height 13 Pt; 24 Pt above and below paragraph, centred). Abstract, abstract, abstract, abstract, abstract, abstract, abstract, abstract, abstract, abstract, abstract, abstract, abstract, abstract, abstract, abstract, abstract, abstract, abstract, abstract, abstract, abstract, abstract, abstract, abstract, abstract, abstract, abstract, abstract, abstract, abstract, abstract, abstract, abstract, abstract, abstract, abstract, abstract, abstract, abstract, abstract, abstract, abstract, abstract, abstract, abstract. 
\end{abstract}

\vspace{24pt}

\section{Heading 1 (Times 18; 24 Pt above; 12 Pt below paragraph; flush left)}

Job areas with on-call service mark a very interesting example of a boundary in which the consequences for unwanted transitions are especially high. The influences of constant interruptibility on mental and cognitive state become obvious. A questionnaire in 2006 and 2010 among Finnish physicians found that sleeping problems and WIF (work interference with family) lead to high distress, low job satisfaction and low work ability. The number of active on-call hours is associated with higher levels of WIF, but not with sleeping problems.
Another study lists symptoms for mental, cognitive and behavioural effects. Table 1 shows a selection of symptoms. The kind of symptoms and their substantial prevalence among workers in on-call service shows the severity of the effects.

\subsection{Heading 2 (Times 14; 18 Pt above; 9 Pt below; flush left)}

Text in Times 12; height 15. No indenting for first paragraph. No indenting for first paragraph. No indenting for first paragraph. No indenting for first paragraph. 

\begin{itemize}
\item Bullet list (Times 12; height 15; indenting 5 mm) 
\item Bullet list (Times 12; height 15; indenting 5 mm) 
\end{itemize}

Subsequent paragraphs should have same font and layout, but with 5 mm indenting in the first line. 

\begin{enumerate}
\item Numbered list (Times 12; height 15; indenting 5 mm)
\item Numbered list (Times 12; height 15; indenting 5 mm)
\end{enumerate}
 
\subsubsection{Heading 3 (Times 12; 12 Pt above; 6 Pt below; flush left)}

Text in Times 12; height 15. No indenting for first paragraph. No indenting for first paragraph. No indenting for first paragraph. No indenting for first paragraph. 

And tables are referenced in the text like this (cf. Table ~\ref{CML_Seminar_Template_tab1}).

\begin{table}
\begin{center}
\begin{tabular}{ |c|c|} 
 \hline
 Key & Value \\
 \hline
 X & A \\
 \hline
\end{tabular}
\end{center}
    \caption{\label{CML_Seminar_Template_tab1} Table caption (Times 10; height 13 Pt; 12 above and below). Leave 12 Pt free above of the figure (use the same format for Tables, but with roman numbering). }
\end{table}

Subsequent paragraphs should have same font and layout, but with 5 mm indenting in the first line. 

\begin{quote}
Quotations with a length of more than 40 words should be put in a separate paragraph (Times 10; height 13; 2 Pt above; 2 Pt below; 5 mm indenting)
\end{quote}


Brands, company names, etc. should be written in capital letters (e.g., \textsc{The Coordinator}). The paper should have a length of about 16 pages. The pages should be numbered (page numbers in Helvetica 8; italic). Use a margin of 3.0 cm on the left and right side, and 2.5 cm on top and bottom. Refer to figures in the text (cf. Figure ~\ref{CML_Seminar_Template_fig1}).


\begin{figure}[htb]
  \begin{center}
%  \includegraphics{CML_Seminar_Template_fig1}
  \end{center}
    \caption{\label{CML_Seminar_Template_fig1} Caption (Times 10; height 13 Pt; 12 above and below). Leave 12 Pt free above of the figure (use the same format for Tables, but with roman numbering).}
\end{figure}

References in the text should have the format [Author Year]. Sources with one author should be referenced as \cite[]{Gro1995}; sources with two authors like this \cite[]{MaVa1984}, and sources with more than two authors like this \cite[]{Ham2002}; if you have the page numbers, indicate them like this \cite[p. 16]{Gro1995} or this for a range \cite[pp. 16-17]{Gro1995}. 

\section{Heading 1 (Times 18; 24 Pt above; 12 Pt below paragraph; flush left)}

The reference list at the end of the paper should be in Times 10; height 13; 5 mm indenting; 1 Pt above; at the end of the paper; sorted by author(s) and year. Some examples: 


\subsection*{Books:} 

\cite[]{Ham2002, MaVa1984}

\subsection*{Journals:}

\cite[]{Gro1995}

\subsection*{On-line Sources:} 

\cite[]{Bal1994}

\subsection*{Conference Proceedings:} 

\cite[]{GrPr2003}

\subsection*{Ph.D. and Master's theses:}

\cite[]{Dou1996}

\subsection*{Reports:} 

\cite[]{BeMa1993}


%\section*{Acknowledgements}
%
%\begin{acknowledgements}
%Acknowledgements. 
%\end{acknowledgements}

\bibliography{CML_Seminar_Template}

\end{document}
